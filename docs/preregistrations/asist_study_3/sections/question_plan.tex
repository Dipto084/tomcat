\chapter{Question-Asking and Plan Inference}
\label{ch:question_plan}
\textbf{Salena Ashton, Stephen Kim, Loren Champlin, Liang Zhang, Clayton Morrison}

\section{Introduction}

As people interact with each other, they display observable behaviors (speech,
body language, expressed emotion, or interaction with the environment) that may
suggest unobservable behaviors (thoughts, feelings, plans, or beliefs). The
process of inferring these unobservable behaviors is called \textit{Theory of
Mind} (ToM) and is often studied through the analysis of body language, spoken natural language, human development, social cultural differences or non-neurotypical human cognition. When humans ask questions to obtain information, they most likely do so with an intent, a plan, or desire.

The term \textit{planning} in Artificial Intelligence research does not have
the same meaning as it would with human ToM. Within AI, a \textit{classical
plan}, also called flat, is a sequence of states and simple actions that an
agent can execute to reach a goal. These plans can be represented with a flat
data structure such as a list. As plans become more complex, require
constraints, or have multiple levels of abstraction within the same plan, they
are best represented by a \textit{hierarchical task network} (HTN), which is a tree of possible plans.
\vspace{15pt}

\begin{enumerate}
    \item How do spoken questions reveal another person’s plan or intent? We will investigate whether verbalized question-asking can infer a human’s ToM
when uttered in a simulated search-and-rescue (SAR) scenario within the
Minecraft environment, as designed by ASIST.
    \item Can those who listen to these questions accurately decide if the plan is simple, sequential, or hierarchical? The results of this investigation will guide further research about how to
infer and best represent a human's plans as either flat or hierarchical. Can
the distinction of such structures improve predictive performance for Automated
Social Intelligence (ASI) agent intervention?
\end{enumerate}

\vspace{15pt}

Existing literature about HTNs, ToM, psycholinguistics and question-answering
are abundant, as are papers about question generation for information
retrieval. Few papers focus on the intersection of ToM for human beings and the HTN mapping of dialog acts. Hawkins and Goodman \citep{hawkins2017you} approach question-asking as evidence of the questioner’s hidden goals. They categorize questions into binary question-answer pairs, pragmatic question-answer pairs where the most socially-acceptable response does not strictly answer the literal question asked, and probabilistic question-answer pairs, where the most probable answer is the expected value of a set of likely answers. Within the SAR scenario of ASIST’s Minecraft environment, human players will
verbalize plans, suggestions, or ideas to each other. Sometimes these plans or
requests are fulfilled and other times they are violated, forgotten or
disregarded. Baldoni et al \citep{baldoni_2018} discuss social interactions
of AI agents with first-order logic; this representation includes
question-asking and the life cycle of agent commitments. This representation can also express action sequence and constraints of actions and plans.



\section{Approach}

We assume that questions have hidden goals and infer plans. As teams ask more questions of each other, human team ToM converges toward cooperative behavior.
We will investigate whether question-asking is associated with \textit{team
planning}, defined as a set of goals, strategies, or tasks that are executed.
We define \textit{coordination} as behaviors and utterances to create a common
plan or strategy. We define \textit{cooperation} as team behaviors that implement an already-agreed up on plan.

To capture hidden goals, inferred plans, and patterns that may represent human
ToM, we will annotate ASIST Study 3 Spiral 2 pilot video observations: 418, 419, 420, 421, 422, and 423.
These six pilot videos are of three distinct missions each for two distinct
teams. Due to the expensive costs of human annotation and strict adherence to grounded theory methodology, this experiment is limited to six videos. 

Two human annotators will code all uttered questions between teammates within the Minecraft SAR scenario.
We use the qualitative coding procedure known as Grounded Theory,
as defined by Corbin and Strauss, \citep{corbin_strauss_2015}. 

More specifically, and as defined by Saldana, \citep{saldana_2021}, we will use a Grounded Theory Process Coding for a state or action across some interval of time. These \textit{grounded-in-data} labels are known as \textit{concept-level} labels, which are the smallest pieces of data that encode a question-asking phenomena of interest.

We will use a Grounded Theory Causal Coding to investigate the connectivity and causality of each concept label to discover possible relationships between presence or absence of team actions, interactions, conditions, and consequences of question-asking. Densely-connected concept labels suggest subcategories and categories. Sparsely-connected labels will not be discarded; they will be used to consider variability within patterns and categories that emerge. In cases where questions have co-reference or other contextual dependencies, only that direct dependency will be coded for local semantic meaning.

\vspace{15pt}

We make the following considerations when creating codes: 

\begin{itemize}
    \item Frequency will not dictate importance, causality, or connectivity of a concept
    \item Labels will be stemmed and minimally normalized 
    \item Capturing the phenomena of question-asking across time, between any subset of a team, between the same team across the two different missions. 
\end{itemize}

\vspace{15pt}

To avoid annotator and researcher biases and any \textit{a priori} belief on
which team ToM strategies may be used, concept and category labels are not
pre-determined. Annotator agreement must reach a Kappa Score of 80\% or higher. This also gives a more solid, grounded analytical meaning to any emergent categories. 

When all concepts, subcategories, categories can reasonably explain the phenomena of the video observations, one or two super-categories, \textit{theories of team plan}, will emerge. We currently assume that a theory of team plan would have greater predictive power and ToM inference potential. 

Future testing of these labels will be on the real data, released after 29 March 2022, in future pre-registrations and experiments. 






\section{Evaluation}

Because of the small sample size of this investigation, we do not evaluate at
this time. Instead, we investigate word frequencies, clustering patterns, and
correlation of annotator-generated labels through data visualization. Below is
a list of possible visualizations we may consider:

\begin{itemize}
    \item Connectivity of concept-level labels: radial diagrams, arc diagrams, matrix diagrams or graph networks
    \item Frequency patterns of words or concept labels (normalized, word count / total number of words in that question): scatterplots or histograms
    \item Correlation of words and labels with time: time series, scatterplots. 
    \item Concept-level subcategorization(s): clustering, PCA (concept labels
      possibly projected onto sub-categorical spaces), or hierarchical visualizations.
\end{itemize}


These visualizations will serve as a preliminary analysis of how uttered questions, with their hidden goals, could map  human ToM to AI planning data structures.

Future measures may include Mann-Whitney U-Tests, t-tests (only if we annotate a large-enough sample), precision and recall of the
concept-level and category patterns to describe the generalizability for real data with no ASI interventions, generalizability for
real data with higher ASI interventions, and the variance of patterns in label categories. Another possible measure, from future research, would be the F1 Score to explain how
well these labels describe observations without ASI interventions, when compared to
high-intervention observations. This future investigation would address the TA3’s Process Measure ASI-M5: Coordinative Communications   to measure teamwork, include additional video observations for real data in Study 3, and continue our investigation of whether human plans and ToM are best represented by classical planning or HTN planning.


